% This is a template for your written document.
%
% To compile using latexmk on the command line, run the following: 
% latexmk -pdf main.tex

\documentclass[12pt]{article}
\usepackage{setspace}
\usepackage{graphicx} % used for includegraphics
\singlespace
\usepackage[left=1in,right=1in,top=1in,bottom=1in]{geometry}

\title{\textbf{Project Topic}}
\author{Cameron Sentieri}

\begin{document}

\maketitle

In a world where technology is the most accessible it's ever been, sitting still has become increasingly difficult. Books are being replaced by e-readers, physical stores are being replaced by online shopping, and in-person communication is being replaced ny social media, texting, and video calls. While these developments offer convenience and connectivity, they also contribute to a constant state of stimulation.  With this rapid advancement in technology, it is important to take time to slow down, get off our devices, connect with others, and with God. Technology has a way of intruding on our daily lives and detracting from activities like pausing, reflecting, praying, and engaging in meaningful conversations[1]. With that being said, my project aims to create a mobile application that encourages users to take breaks from their devices and engage in meaningful activities that promote mental well-being and spiritual growth. This project matters because while concerns about distraction and spiritual attentiveness are frequently discussed, there is limited research that examines how everyday technology use shapes these experiences. This project seeks to give insight on the beneficial and detrimental effects of technology on spritual life, as well as asking the question of whether intentional technological design can support attentiveness rather than detract from it.
It is important to note that technology is not inherently bad or harmful to spiritual life, infact it has been helpful in spreading the message of the cross and the gosepl by providing more information to more people in a shorter amount of time[2]. 



In your proposal, you will provide justification on why your project matters based on work which has been done in this area, using in-line citations~\cite{clrsAlgorithms} to refer to existing works. Include what your project will accomplish and how your software will function. 

When you use an image, such as in Figure~\ref{fig:method}, refer to it in the text.

\begin{figure}[h]
\begin{center}
\includegraphics[scale=0.7]{methodology.png}
\caption{Archie}
\label{fig:method}       % Give a unique label
\end{center}
\end{figure}


\newpage
\section*{Appendix}
A concise list of features / user stories in the order in which they will be built. A few examples are below to demonstrate the expected scope and level of granularity; you will have more features than this.
\begin{itemize}
	\item Default picture display on web application.
	\item On a button-click, user can separate the image into foreground and background.
	\item User can select a picture from their desktop.
	\item Selected picture displays on the web application.
\end{itemize}


\bibliographystyle{acm}
\bibliography{bibliography.bib}


\end{document}
