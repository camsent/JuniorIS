% This is a template for your written document.
%
% To compile using latexmk on the command line, run the following: 
% latexmk -pdf main.tex

\documentclass[12pt]{article}
\usepackage{setspace}
\usepackage{graphicx} % used for includegraphics
\singlespace
\usepackage[left=1in,right=1in,top=1in,bottom=1in]{geometry}

\title{\textbf{Project Topic}}
\author{Cameron Sentieri}

\begin{document}

\maketitle

In a world where technology is the most accessible it's ever been, sitting still has become increasingly difficult. Books are being replaced by e-readers, physical stores are being replaced by online shopping, and in-person communication is being replaced ny social media, texting, and video calls. While these developments offer convenience and connectivity, they also contribute to a constant state of stimulation.  With this rapid advancement in technology, it is important to take time to slow down, get off our devices, connect with others, and with God. Technology has a way of intruding on our daily lives and detracting from activities like pausing, reflecting, praying, and engaging in meaningful conversations~\cite{10.1145/3538706}. It is important to note that technology is not inherently bad or harmful to spiritual life, infact it has been helpful in spreading the message of the cross and the gosepl by providing more information to more people in a shorter amount of time~\cite{sims2001effect}. With that being said, this project aims to ask how technology can be beneficial, as well as detrimental to someones relationship with God. The mobile application that will be built to support the research will encourage users to take breaks from their devices and engage in meaningful activities that promote mental well-being and spiritual growth. This research matters because while concerns about distraction and spiritual attentiveness are frequently discussed, there is limited research that examines how everyday technology use shapes these experiences. While research on digital Christian practices and online worhsip has begun to emerge, much of this work focuses on communal or institutional forms of worship rather than everyday habits of attentiveness and silence in personal spritual life~\cite{10.1145/3491101.3519856}. This work seeks to give insight on the beneficial and detrimental effects of technology on spritual life, as well as asking the question of whether intentional technological design can support attentiveness rather than detract from it. In order to gather data for this project, professional interviews with pastors will be done to see the effects that technology has had on their spiritual lives. There will also be surveys sent out to a variety of people to get a broader perspective on the everyday Christian. In addition to gathering data, the study will include the implementation of a minimalistic software application focused on protecting periods of stillness. Unlike existing faith based applications that often bombard users with notifications and spiritual content, the apps focus is to reduce digital stimulation by limiting notifications and interruptions during user-defind periods of quiet. This application functions not as a guide for prayer, but as a tool to create space for users to engage in their own spiritual practices without technological distractions. The application will also serve as a research tool, allowing participants to engage with intentional periods of stillness over time. Its effectiveness will be evaluated through pre and post-study survey responses that will assess changes in users' attentiveness, frequency of distraction, and comfort with silence during spiritual practices. Rather than measuring spiritual outcomes directly, the study focuses on changes in users' attentional habits that many Christians find important for their spiritual life.
This study is limited by its reliance on self-reported data and a relatively small participant pool. Additionally, spiritual attentiveness is a deeply personal experience that cannot be fully captured through quantitative measures. Despite these limitations, the project offers meaningful insight into how technology can be intentionally designed to support stillness and reflection. By combining research with software design, this work contributes to broader conversations about attention, faith, and ethical technology design.






When you use an image, such as in Figure~\ref{fig:method}, refer to it in the text.

\begin{figure}[h]
\begin{center}
\includegraphics[scale=0.7]{methodology.png}
\caption{Archie}
\label{fig:method}       % Give a unique label
\end{center}
\end{figure}


\newpage
\section*{Appendix}
A concise list of features / user stories in the order in which they will be built. A few examples are below to demonstrate the expected scope and level of granularity; you will have more features than this.
\begin{itemize}
	\item Intentional Stillness Windows: Users schedule periods of quiet explicity for spiritual attentiveness, reflection, or prayer. Session length, frequency, and interruptions are logged for study.
	\item Smart Notification Management: Selectively blocks nonessential notifications during stillness windows while allowing critical alerts. Tracks which notifications are silenced or bypassed, creating quantifiable data on digital interruptions. 
	\item 
	\item Selected picture displays on the web application.
\end{itemize}


\bibliographystyle{acm}
\bibliography{bibliography.bib}


\end{document}
